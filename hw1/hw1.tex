\documentclass[9pt,twocolumn,twoside,]{pnas-new}

% Use the lineno option to display guide line numbers if required.
% Note that the use of elements such as single-column equations
% may affect the guide line number alignment.


\usepackage[T1]{fontenc}
\usepackage[utf8]{inputenc}
% Pandoc syntax highlighting
\usepackage{color}
\usepackage{fancyvrb}
\newcommand{\VerbBar}{|}
\newcommand{\VERB}{\Verb[commandchars=\\\{\}]}
\DefineVerbatimEnvironment{Highlighting}{Verbatim}{commandchars=\\\{\}}
% Add ',fontsize=\small' for more characters per line
\newenvironment{Shaded}{}{}
\newcommand{\AlertTok}[1]{\textcolor[rgb]{1.00,0.00,0.00}{\textbf{#1}}}
\newcommand{\AnnotationTok}[1]{\textcolor[rgb]{0.38,0.63,0.69}{\textbf{\textit{#1}}}}
\newcommand{\AttributeTok}[1]{\textcolor[rgb]{0.49,0.56,0.16}{#1}}
\newcommand{\BaseNTok}[1]{\textcolor[rgb]{0.25,0.63,0.44}{#1}}
\newcommand{\BuiltInTok}[1]{#1}
\newcommand{\CharTok}[1]{\textcolor[rgb]{0.25,0.44,0.63}{#1}}
\newcommand{\CommentTok}[1]{\textcolor[rgb]{0.38,0.63,0.69}{\textit{#1}}}
\newcommand{\CommentVarTok}[1]{\textcolor[rgb]{0.38,0.63,0.69}{\textbf{\textit{#1}}}}
\newcommand{\ConstantTok}[1]{\textcolor[rgb]{0.53,0.00,0.00}{#1}}
\newcommand{\ControlFlowTok}[1]{\textcolor[rgb]{0.00,0.44,0.13}{\textbf{#1}}}
\newcommand{\DataTypeTok}[1]{\textcolor[rgb]{0.56,0.13,0.00}{#1}}
\newcommand{\DecValTok}[1]{\textcolor[rgb]{0.25,0.63,0.44}{#1}}
\newcommand{\DocumentationTok}[1]{\textcolor[rgb]{0.73,0.13,0.13}{\textit{#1}}}
\newcommand{\ErrorTok}[1]{\textcolor[rgb]{1.00,0.00,0.00}{\textbf{#1}}}
\newcommand{\ExtensionTok}[1]{#1}
\newcommand{\FloatTok}[1]{\textcolor[rgb]{0.25,0.63,0.44}{#1}}
\newcommand{\FunctionTok}[1]{\textcolor[rgb]{0.02,0.16,0.49}{#1}}
\newcommand{\ImportTok}[1]{#1}
\newcommand{\InformationTok}[1]{\textcolor[rgb]{0.38,0.63,0.69}{\textbf{\textit{#1}}}}
\newcommand{\KeywordTok}[1]{\textcolor[rgb]{0.00,0.44,0.13}{\textbf{#1}}}
\newcommand{\NormalTok}[1]{#1}
\newcommand{\OperatorTok}[1]{\textcolor[rgb]{0.40,0.40,0.40}{#1}}
\newcommand{\OtherTok}[1]{\textcolor[rgb]{0.00,0.44,0.13}{#1}}
\newcommand{\PreprocessorTok}[1]{\textcolor[rgb]{0.74,0.48,0.00}{#1}}
\newcommand{\RegionMarkerTok}[1]{#1}
\newcommand{\SpecialCharTok}[1]{\textcolor[rgb]{0.25,0.44,0.63}{#1}}
\newcommand{\SpecialStringTok}[1]{\textcolor[rgb]{0.73,0.40,0.53}{#1}}
\newcommand{\StringTok}[1]{\textcolor[rgb]{0.25,0.44,0.63}{#1}}
\newcommand{\VariableTok}[1]{\textcolor[rgb]{0.10,0.09,0.49}{#1}}
\newcommand{\VerbatimStringTok}[1]{\textcolor[rgb]{0.25,0.44,0.63}{#1}}
\newcommand{\WarningTok}[1]{\textcolor[rgb]{0.38,0.63,0.69}{\textbf{\textit{#1}}}}

% tightlist command for lists without linebreak
\providecommand{\tightlist}{%
  \setlength{\itemsep}{0pt}\setlength{\parskip}{0pt}}




\templatetype{pnasmathematics}  % Choose template

\title{Problem Set 1}

\author[]{Carlos Enrique Lezama Jacinto}

  \affil[]{Instituto Tecnológico Autónomo de México}


% Please give the surname of the lead author for the running footer
\leadauthor{}

% Please add here a significance statement to explain the relevance of your work
\significancestatement{}


\authorcontributions{}



\correspondingauthor{\textsuperscript{} My solutions to the first
problem set in Advanced Microeconometrics (ECO -- 20513).}

% Keywords are not mandatory, but authors are strongly encouraged to provide them. If provided, please include two to five keywords, separated by the pipe symbol, e.g:


\begin{abstract}

\end{abstract}

\dates{This manuscript was compiled on \today}
\doi{\url{www.pnas.org/cgi/doi/10.1073/pnas.XXXXXXXXXX}}

\begin{document}

% Optional adjustment to line up main text (after abstract) of first page with line numbers, when using both lineno and twocolumn options.
% You should only change this length when you've finalised the article contents.
\verticaladjustment{-2pt}



\maketitle
\thispagestyle{firststyle}
\ifthenelse{\boolean{shortarticle}}{\ifthenelse{\boolean{singlecolumn}}{\abscontentformatted}{\abscontent}}{}

% If your first paragraph (i.e. with the \dropcap) contains a list environment (quote, quotation, theorem, definition, enumerate, itemize...), the line after the list may have some extra indentation. If this is the case, add \parshape=0 to the end of the list environment.

\acknow{}

\newcommand{\ind}{\perp\!\!\!\!\perp}

\hypertarget{the-model}{%
\section*{The Model}\label{the-model}}
\addcontentsline{toc}{section}{The Model}

Assume that the economic model generating the data for potential
outcomes is of the form:

\begin{align*}
Y_1 &= \alpha + \varphi + U_1, \\
Y_0 &= \alpha + U_0,
\end{align*}

where \(U_1\) and \(U_0\) represent the unobservables in the potential
outcome equations and \(\varphi\) represents the benefit associated with
the treatment (\(D = 1\)). Individuals decide whether or not to receive
the treatment (\(D = 1\) or \(D = 0\)) based on a latent variable \(I\):

\[
I = Z \gamma + V,
\]

where \(Z\) and \(V\) represent observables and unobservables,
respectively. Thus, we can define a binary variable \(D\) indicating
treatment status,

\[
D = \mathds{1}\left[ I \geq 0\right].
\]

Finally, we assume that the error terms in the model are not independent
even conditioning on the observables,
i.e.~\(U_1 \not\perp\!\!\!\!\perp U_0 \not\perp\!\!\!\!\perp V \mid Z\),
but \((U_1, U_0, V) \perp\!\!\!\!\perp Z\).

\hypertarget{treatment-parameters}{%
\section{Treatment Parameters}\label{treatment-parameters}}

Let \(\beta = Y_1 - Y_0 = \varphi + U_1 - U_0\) such that, in regression
notation, \(Y = \alpha + \beta D + \varepsilon\) where
\(\varepsilon = U_0\).

\hypertarget{average-treatment-effect}{%
\subsection*{Average Treatment Effect}\label{average-treatment-effect}}
\addcontentsline{toc}{subsection}{Average Treatment Effect}

\begin{align*}
ATE &= E \left[ \beta \right] \\
&= E \left[ \varphi + U_1 - U_0 \right] \\
&= \varphi
\end{align*}

\hypertarget{treatment-on-the-treated}{%
\subsection*{Treatment on the Treated}\label{treatment-on-the-treated}}
\addcontentsline{toc}{subsection}{Treatment on the Treated}

\begin{align*}
TT &= E \left[ \beta \mid D = 1 \right] \\
&= E \left[ \varphi + U_1 - U_0 \mid Z \gamma + V \geq 0 \right] \\
&= \varphi + E \left[ U_1 - U_0 \mid V \geq - Z \gamma \right]
\end{align*}

\hypertarget{treatment-on-the-untreated}{%
\subsection*{Treatment on the
Untreated}\label{treatment-on-the-untreated}}
\addcontentsline{toc}{subsection}{Treatment on the Untreated}

\begin{align*}
TUT &= E \left[ \beta \mid D = 0 \right] \\
&= E \left[ \varphi + U_1 - U_0 \mid Z \gamma + V < 0 \right] \\
&= \varphi + E \left[ U_1 - U_0 \mid V < - Z \gamma \right]
\end{align*}

\hypertarget{marginal-treatment-effect}{%
\subsection*{Marginal Treatment
Effect}\label{marginal-treatment-effect}}
\addcontentsline{toc}{subsection}{Marginal Treatment Effect}

\begin{align*}
MTE &= E \left[ \beta \mid I = 0,\ V = v \right] \\
&= E \left[ \varphi + U_1 - U_0 \mid I = 0,\ V = v \right] \\
&= \varphi + E \left[ U_1 - U_0 \mid Z \gamma = - V,\ V = v \right]
\end{align*}

\hypertarget{instrumental-variables}{%
\subsection*{Instrumental Variables}\label{instrumental-variables}}
\addcontentsline{toc}{subsection}{Instrumental Variables}

\[
\hat{\beta}_{\text{IV}} (J(Z)) = \frac{\text{Cov}(J(Z), Y)}{\text{Cov}(J(Z), D)} \overset{p}{\longrightarrow} \beta
\]

\hypertarget{ordinary-least-squares}{%
\subsection*{Ordinary Least Squares}\label{ordinary-least-squares}}
\addcontentsline{toc}{subsection}{Ordinary Least Squares}

\[
\hat{\beta}_{\text{OLS}} = \frac{\text{Cov}(Y, D)}{\text{Var}(D)} \implies \hat{\beta}_{\text{OLS}} = \left( D^T D \right)^{-1} D^T Y
\]

\hypertarget{local-average-treatment-effect}{%
\subsection*{Local Average Treatment
Effect}\label{local-average-treatment-effect}}
\addcontentsline{toc}{subsection}{Local Average Treatment Effect}

\begin{align*}
LATE &= E \left[ \beta \mid D(z) = 0, D(z') = 1 \right] \\
&= E \left[ \beta \mid z \gamma < - V \leq z' \gamma \right] \\
&= \varphi + E \left[ U_1 - U_0 \mid - z' \gamma \leq V < - z \gamma \right]
\end{align*}

\hypertarget{some-closed-form-expressions}{%
\section{Some closed form
expressions}\label{some-closed-form-expressions}}

Suppose that the error terms in the model have the following structure:

\begin{align*}
U_1 &= \sigma_1 \epsilon, \\
U_0 &= \sigma_0 \epsilon, \\
V &= \sigma^*_V \epsilon, \\
\epsilon &\sim \mathcal{N} (0, 1).
\end{align*}

So, we can say that

\begin{align*}
TT &= \varphi + E \left[ U_1 - U_0 \mid V \geq - Z \gamma \right] \\
&= \varphi + E \left[ \epsilon (\sigma_1 - \sigma_0) \mid \epsilon \geq \frac{- Z \gamma}{\sigma^*_V} \right] \\
&= \varphi + (\sigma_1 - \sigma_0) E \left[ \epsilon \mid \epsilon \geq \frac{- Z \gamma}{\sigma^*_V} \right] \\
&= \varphi + (\sigma_1 - \sigma_0) \frac{\phi \left( - Z \gamma / \sigma^*_V \right)}{1 - \Phi \left( - Z \gamma / \sigma^*_V \right) },
\end{align*}

and

\begin{align*}
TUT &= \varphi + E \left[ U_1 - U_0 \mid V < - Z \gamma \right] \\
&= \varphi + E \left[ \epsilon (\sigma_1 - \sigma_0) \mid \epsilon < \frac{- Z \gamma}{\sigma^*_V} \right] \\
&= \varphi + (\sigma_1 - \sigma_0) E \left[ \epsilon \mid \epsilon < \frac{- Z \gamma}{\sigma^*_V} \right] \\
&= \varphi - (\sigma_1 - \sigma_0) \frac{\phi \left( - Z \gamma / \sigma^*_V \right)}{\Phi \left( - Z \gamma / \sigma^*_V \right) } \\
&= \varphi + (\sigma_0 - \sigma_1) \frac{\phi \left( - Z \gamma / \sigma^*_V \right)}{\Phi \left( - Z \gamma / \sigma^*_V \right) }.
\end{align*}

\hypertarget{parametrization-1}{%
\section{Parametrization 1}\label{parametrization-1}}

Now, to add more structure to the problem, suppose that
\(Z = (1, Z_1, Z_2)\), and \(\gamma = (\gamma_0, \gamma_1, \gamma_2)\).
Also, suppose that

\begin{align*}
\gamma_0 = 0.2, && \gamma_1 = 0.3, && \gamma_2 = 0.1, \\
\sigma_1 = 0.012, && \sigma_0 = 0.05, && \sigma^*_V = 1, \\
\alpha = 0.02, && \varphi = 0.2.
\end{align*}

Finally,

\begin{align*}
Z_1 &\sim \mathcal{N}(-1, 9), \\
Z_2 &\sim \mathcal{N} (1, 9),
\end{align*}

where \(Z_1 \perp\!\!\!\!\perp Z_2\). Since we only observe either
\(Y_1\) or \(Y_0\), the econometrician observes the outcome described as
follows:

\[
Y = DY_1 + (1 - D)Y_0.
\]

\newpage

\begin{Shaded}
\begin{Highlighting}[]
\FunctionTok{set.seed}\NormalTok{(}\DecValTok{1234}\NormalTok{)}

\NormalTok{N }\OtherTok{\textless{}{-}} \DecValTok{5000} \CommentTok{\# random sample observations}

\NormalTok{alpha }\OtherTok{\textless{}{-}} \FloatTok{0.02}
\NormalTok{phi }\OtherTok{\textless{}{-}} \FloatTok{0.2}

\NormalTok{g }\OtherTok{\textless{}{-}} \FunctionTok{matrix}\NormalTok{(}\FunctionTok{c}\NormalTok{(}\FloatTok{0.2}\NormalTok{, }\FloatTok{0.3}\NormalTok{, }\FloatTok{0.1}\NormalTok{))}
\NormalTok{sigma1 }\OtherTok{\textless{}{-}} \FloatTok{0.012}
\NormalTok{sigma0 }\OtherTok{\textless{}{-}} \FloatTok{0.05}
\NormalTok{sigmaV }\OtherTok{\textless{}{-}} \DecValTok{1}

\NormalTok{eps }\OtherTok{\textless{}{-}} \FunctionTok{rnorm}\NormalTok{(N)}

\NormalTok{U1 }\OtherTok{\textless{}{-}}\NormalTok{ sigma1 }\SpecialCharTok{*}\NormalTok{ eps}
\NormalTok{U0 }\OtherTok{\textless{}{-}}\NormalTok{ sigma0 }\SpecialCharTok{*}\NormalTok{ eps}
\NormalTok{V }\OtherTok{\textless{}{-}}\NormalTok{ sigmaV }\SpecialCharTok{*}\NormalTok{ eps}

\NormalTok{Z }\OtherTok{\textless{}{-}} \FunctionTok{cbind}\NormalTok{(}\FunctionTok{rep}\NormalTok{(}\DecValTok{1}\NormalTok{, N),}
           \FunctionTok{rnorm}\NormalTok{(N, }\SpecialCharTok{{-}}\DecValTok{1}\NormalTok{, }\DecValTok{9}\NormalTok{),}
           \FunctionTok{rnorm}\NormalTok{(N, }\DecValTok{1}\NormalTok{, }\DecValTok{9}\NormalTok{))}

\NormalTok{I }\OtherTok{\textless{}{-}}\NormalTok{ (Z }\SpecialCharTok{\%*\%}\NormalTok{ g) }\SpecialCharTok{+}\NormalTok{ V}
\NormalTok{D }\OtherTok{\textless{}{-}} \FunctionTok{ifelse}\NormalTok{(I }\SpecialCharTok{\textgreater{}=} \DecValTok{0}\NormalTok{, }\DecValTok{1}\NormalTok{, }\DecValTok{0}\NormalTok{)}

\NormalTok{Y1 }\OtherTok{\textless{}{-}}\NormalTok{ alpha }\SpecialCharTok{+}\NormalTok{ phi }\SpecialCharTok{+}\NormalTok{ U1}
\NormalTok{Y0 }\OtherTok{\textless{}{-}}\NormalTok{ alpha }\SpecialCharTok{+}\NormalTok{ U0}

\NormalTok{Y }\OtherTok{\textless{}{-}}\NormalTok{ (D }\SpecialCharTok{*}\NormalTok{ Y1) }\SpecialCharTok{+}\NormalTok{ ((}\DecValTok{1} \SpecialCharTok{{-}}\NormalTok{ D) }\SpecialCharTok{*}\NormalTok{ Y0)}

\NormalTok{df }\OtherTok{\textless{}{-}} \FunctionTok{data.frame}\NormalTok{(}\StringTok{\textquotesingle{}beta\textquotesingle{}} \OtherTok{=}\NormalTok{ Y1 }\SpecialCharTok{{-}}\NormalTok{ Y0,}
                 \StringTok{\textquotesingle{}decision\textquotesingle{}} \OtherTok{=}\NormalTok{ D,}
                 \StringTok{\textquotesingle{}y1\textquotesingle{}} \OtherTok{=}\NormalTok{ Y1,}
                 \StringTok{\textquotesingle{}y0\textquotesingle{}} \OtherTok{=}\NormalTok{ Y0,}
                 \StringTok{\textquotesingle{}net utility\textquotesingle{}} \OtherTok{=}\NormalTok{ I)}

\NormalTok{ATE }\OtherTok{\textless{}{-}}\NormalTok{ df }\SpecialCharTok{\%\textgreater{}\%}
    \FunctionTok{pull}\NormalTok{(beta) }\SpecialCharTok{\%\textgreater{}\%} 
    \FunctionTok{mean}\NormalTok{()}

\NormalTok{TT }\OtherTok{\textless{}{-}}\NormalTok{ df }\SpecialCharTok{\%\textgreater{}\%}
    \FunctionTok{filter}\NormalTok{(decision }\SpecialCharTok{==} \DecValTok{1}\NormalTok{) }\SpecialCharTok{\%\textgreater{}\%}
    \FunctionTok{pull}\NormalTok{(beta) }\SpecialCharTok{\%\textgreater{}\%}
    \FunctionTok{mean}\NormalTok{()}

\NormalTok{TUT }\OtherTok{\textless{}{-}}\NormalTok{ df }\SpecialCharTok{\%\textgreater{}\%}
    \FunctionTok{filter}\NormalTok{(decision }\SpecialCharTok{==} \DecValTok{0}\NormalTok{) }\SpecialCharTok{\%\textgreater{}\%}
    \FunctionTok{pull}\NormalTok{(beta) }\SpecialCharTok{\%\textgreater{}\%}
    \FunctionTok{mean}\NormalTok{()}

\NormalTok{MTE }\OtherTok{\textless{}{-}}\NormalTok{ df }\SpecialCharTok{\%\textgreater{}\%}
    \FunctionTok{filter}\NormalTok{(}\FunctionTok{abs}\NormalTok{(net.utility }\SpecialCharTok{{-}}\NormalTok{ V) }\SpecialCharTok{\textless{}=} \FloatTok{0.01}\NormalTok{) }\SpecialCharTok{\%\textgreater{}\%}
    \FunctionTok{pull}\NormalTok{(beta) }\SpecialCharTok{\%\textgreater{}\%}
    \FunctionTok{mean}\NormalTok{()}
\end{Highlighting}
\end{Shaded}

\begin{center}\includegraphics[width=0.85\linewidth,]{hw1_files/figure-latex/unnamed-chunk-2-1} \end{center}

\hypertarget{parametrization-2}{%
\section{Parametrization 2}\label{parametrization-2}}

\hypertarget{latez_1--2-z_1-1}{%
\subsection{\texorpdfstring{\(LATE(z_1 = -2, z'_1 = 1)\)}{LATE(z\_1 = -2, z'\_1 = 1)}}\label{latez_1--2-z_1-1}}

\begin{Shaded}
\begin{Highlighting}[]
\FunctionTok{print}\NormalTok{(}\DecValTok{1234}\NormalTok{)}
\end{Highlighting}
\end{Shaded}

\begin{verbatim}
## [1] 1234
\end{verbatim}

\hypertarget{latez_2-0-z_2-2}{%
\subsection{\texorpdfstring{\(LATE(z_2 = 0, z'_2 = 2)\)}{LATE(z\_2 = 0, z'\_2 = 2)}}\label{latez_2-0-z_2-2}}

\begin{Shaded}
\begin{Highlighting}[]
\FunctionTok{print}\NormalTok{(}\DecValTok{1234}\NormalTok{)}
\end{Highlighting}
\end{Shaded}

\begin{verbatim}
## [1] 1234
\end{verbatim}

\hypertarget{ivz_1}{%
\subsection{\texorpdfstring{\(IV(Z_1)\)}{IV(Z\_1)}}\label{ivz_1}}

\begin{Shaded}
\begin{Highlighting}[]
\NormalTok{beta.IV.Z1 }\OtherTok{\textless{}{-}} \FunctionTok{cov}\NormalTok{(Z[, }\DecValTok{2}\NormalTok{], Y) }\SpecialCharTok{/} \FunctionTok{cov}\NormalTok{(Z[, }\DecValTok{2}\NormalTok{], D)}
\end{Highlighting}
\end{Shaded}

\(\hat{\beta}_{\text{IV}} (Z_1) = 0.2032\).

\hypertarget{ivz_2}{%
\subsection{\texorpdfstring{\(IV(Z_2)\)}{IV(Z\_2)}}\label{ivz_2}}

\begin{Shaded}
\begin{Highlighting}[]
\NormalTok{beta.IV.Z2 }\OtherTok{\textless{}{-}} \FunctionTok{cov}\NormalTok{(Z[, }\DecValTok{3}\NormalTok{], Y) }\SpecialCharTok{/} \FunctionTok{cov}\NormalTok{(Z[, }\DecValTok{3}\NormalTok{], D)}
\end{Highlighting}
\end{Shaded}

\(\hat{\beta}_{\text{IV}} (Z_2) = 0.2046\).

\hypertarget{ols}{%
\subsection{\texorpdfstring{\(OLS\)}{OLS}}\label{ols}}

\begin{Shaded}
\begin{Highlighting}[]
\NormalTok{beta.OLS }\OtherTok{\textless{}{-}} \FunctionTok{cov}\NormalTok{(Y, D) }\SpecialCharTok{/} \FunctionTok{var}\NormalTok{(D)}
\end{Highlighting}
\end{Shaded}

\(\hat{\beta}_{\text{OLS}} = 0.2186\).

\hypertarget{gmm}{%
\section{GMM}\label{gmm}}

\hypertarget{likelihood-function}{%
\section{Likelihood Function}\label{likelihood-function}}

\hypertarget{discussion}{%
\section{Discussion}\label{discussion}}

\hypertarget{maximum-likelihood}{%
\section{Maximum Likelihood}\label{maximum-likelihood}}

\showmatmethods
\showacknow
\pnasbreak



% Bibliography
% \bibliography{pnas-sample}

\end{document}
